\documentclass{beamer}

\title{Übungsblatt 2: Fourier-Transformation und Parallelisierung}
\author{Fabian Sponholz}
\date{\today}

\begin{document}

\frame{\titlepage}

\section{Einführung}
\begin{frame}{Einführung}
    \begin{itemize}
        \item Ziel des Projekts: Implementierung und Analyse verschiedener Ansätze zur Fourier-Transformation
        \item Vergleich von sequenzieller, multi-threaded und OpenCL-beschleunigter Implementierung
    \end{itemize}
\end{frame}

\section{Lösung Aufgabe 3: Multi-Threading}
\begin{frame}{Lösung Aufgabe 3: Multi-Threading}
    \begin{itemize}
        \item Parallelisierte Version des Cooley-Tukey FFT-Algorithmus mit C++ und Multi-Threading
        \item Nutzung der Bibliotheken \texttt{thread} und \texttt{mutex}
        \item Erheblicher Speed-Up festgestellt: $\sim 3.5 \times$ auf 4-Kern-CPU
    \end{itemize}
\end{frame}

\section{Lösung Aufgabe 4: OpenCL}
\begin{frame}{Lösung Aufgabe 4: OpenCL}
    \begin{itemize}
        \item Implementierung der Diskreten Fourier-Transformation (DFT) mit OpenCL
        \item Herausforderungen und Lösungsansätze:
        \begin{itemize}
            \item Rekursion in OpenCL nicht möglich, Umstellung auf iterative DFT
            \item Initiale Probleme mit komplexem Kernel und instabilen Ergebnissen
            \item Vereinfachter Kernel brachte deterministische und plausible Ergebnisse
        \end{itemize}
    \end{itemize}
\end{frame}

\section{Experimentelle Tests}
\begin{frame}{Experimentelle Tests}
    \begin{itemize}
        \item Testdateien: Stille, 420 Hz, Oktavensprung, Zufällige Frequenzen
        \item Ergebnisse bei allen Implementierungen konsistent:
        \begin{itemize}
            \item Stille: Keine Frequenzen erkannt
            \item 420 Hz: Genauigkeit der Frequenzerkennung
            \item Oktavensprung: Erkennung beider Frequenzen
            \item Zufällige Frequenzen: Erfolgreiche Erkennung mit angepassten Thresholds
        \end{itemize}
    \end{itemize}
\end{frame}

\section{Laufzeitanalyse und Vergleich}
\begin{frame}{Laufzeitanalyse und Vergleich}
    \begin{itemize}
        \item Testsysteme:
        \begin{itemize}
            \item System 1: Intel i5-2500K, NVIDIA GTX 1650, 4 GB VRAM
            \item System 2: Intel i5-10400F, Intel Arc A770, 16 GB VRAM
        \end{itemize}
        \item Laufzeitanalyse auf verschiedenen Parametern (Blockgröße, Schrittweite)
    \end{itemize}
\end{frame}

\section{Laufzeiten Testsystem 1}
\begin{frame}{Laufzeiten Testsystem 1}
    \begin{table}[]
        \centering
        \begin{tabular}{|c|c|c|c|c|}
            \hline
            \textbf{Programm} & \textbf{512/2} & \textbf{Speedup} & \textbf{512/4} & \textbf{Speedup} \\ \hline
            Base & 538 s & 1 & 274 s & 1 \\ \hline
            Threads & 157 s & 3.43 & 77 s & 3.56 \\ \hline
            OpenCL & n/a & n/a & 347 s & 0.79 \\ \hline
        \end{tabular}
    \end{table}
    \begin{table}[]
        \centering
        \begin{tabular}{|c|c|c|}
            \hline
            \textbf{Programm} & \textbf{512/8} & \textbf{Speedup} \\ \hline
            Base & 135 s & 1 \\ \hline
            Threads & 38 s & 3.55 \\ \hline
            OpenCL & 6.5 s & 20.77 \\ \hline
        \end{tabular}
    \end{table}
\end{frame}

\section{Laufzeiten Testsystem 2}
\begin{frame}{Laufzeiten Testsystem 2}
    \begin{table}[]
        \centering
        \begin{tabular}{|c|c|c|c|c|}
            \hline
            \textbf{Programm} & \textbf{512/2} & \textbf{Speedup} & \textbf{512/4} & \textbf{Speedup} \\ \hline
            Base & 367 s & 1 & 190 s & 1 \\ \hline
            Threads & 56.8 s & 6.46 & 28.5 s & 6.67 \\ \hline
            OpenCL & n/a & n/a & n/a & n/a \\ \hline
        \end{tabular}
    \end{table}
    \begin{table}[]
        \centering
        \begin{tabular}{|c|c|c|c|c|}
            \hline
            \textbf{Programm} & \textbf{512/8} & \textbf{Speedup} & \textbf{256/1} & \textbf{Speedup} \\ \hline
            Base & 92 s & 1 & 332 s & 1 \\ \hline
            Threads & 14.3 s & 6.43 & 53,0 s & 6,26 \\ \hline
            OpenCL & 2.29 s & 40.17 & 5,98 & 55,52\\ \hline
        \end{tabular}
    \end{table}
\end{frame}

\section{Zusammenfassung}
\begin{frame}{Zusammenfassung}
    \begin{itemize}
        \item Multi-Threading bietet stabilen Speed-Up, abhängig von der Anzahl der CPU-Kerne
        \item OpenCL zeigt großes Potenzial für Beschleunigung, limitiert durch VRAM und Implementierungsdetails
        \item Weitere Optimierungsmöglichkeiten bei der OpenCL-Implementierung vorhanden
    \end{itemize}
\end{frame}

\end{document}
